\documentclass[twocolumn]{aastex631}
\usepackage{showyourwork}
\usepackage{amsfonts,amssymb,amsmath}

\begin{document}

\title{Constraining gravitational wave amplitude birefringence with GWTC-3}

\author{Thomas C.K. Ng}
\email{thomas.ng@link.cuhk.edu.hk}

\date{\today}

\begin{abstract}
    
\end{abstract}

\section{Points}

\begin{enumerate}
    \item Introduction
    \begin{enumerate}
        \item Motivation
        \begin{enumerate}
            \item GR broke down at some scale, find a new theory
            \item CS: beyond-GR theory
            \item test CS property (GW amplitude birefringence)
            \item GW have 2 polarisations
            \item GR: polarisation ratio depends on iota
        \end{enumerate}
    \end{enumerate}
    \item Method
    \begin{enumerate}
        \item
    \end{enumerate}
    \item Results
    \begin{enumerate}
        \item GW150914
        \begin{enumerate}
            \item
        \end{enumerate}
        \item GWTC-3
        \begin{enumerate}
            \item
        \end{enumerate}
    \end{enumerate}
    \item Discussion
    \begin{enumerate}
        \item Limitation
        \begin{enumerate}
            \item orientation of detector (polarisations)
            \item sensitivity of detector
        \end{enumerate}
        \item Future Study
        \begin{enumerate}
            \item Data from better detector
        \end{enumerate}
    \end{enumerate}
\end{enumerate}

\section{Everything}
Since Einstein proposed his theory of general relativity (GR), it was tested in a wide range of length scales.
After a century, we know that GR do not agree with quantum theories at some length scale.
To unify both theories, we need to study the possibility of different beyond-GR theories.
Some beyond-GR theories such as Chern-Simons theory (CS) suggested that gravitational waves (GWs) amplitude birefringence is a property of space,
which GR does not included.

GW consist of two linear polarizations (i.e. $+$ and $\times$) similar to electromagnetic waves,
which could be transformed into two circular polarizations (i.e. left and right) by $h_{\mathrm{R}, \mathrm{L}} = h_+ \pm i h_\times$.
In GR, the amplitude ratio of the left and right polarisations only depends on $\iota$,
the angle between our line of sight and the orbital angular momentum of the binary black holes (BBH).
\begin{equation}
    \left(\frac{h_\mathrm{L}}{h_\mathrm{R}}\right)_\mathrm{GR}=\left(\frac{1-\cos\iota}{1+\cos\iota}\right)^2\,,
\end{equation}where $h_L$ and $h_R$ are the amplitude of left and right polarizations of the GWs respectively.

For other studies on the birefringence property such as \citet{Maria_2021}, the amplitude ratio depends on not only $\iota$,
but also the distance to the merger and the strength of the birefringence.
\begin{equation}
    \left(\frac{h_\mathrm{L_{obs}}}{h_\mathrm{R_{obs}}}\right)_\mathrm{Biref}^\mathrm{old}=\frac{e^{d_C\widetilde{\kappa}}\left(1-\cos\iota\right)^2}{e^{-d_C\widetilde{\kappa}}\left(1+\cos\iota\right)^2}\,,
\end{equation}where $d_C$ is the comoving distance to the merger, $\widetilde{\kappa}$ is the opacity parameter that represent
the strength of the birefringence with units of $L^{-1}$, and $h_\mathrm{L_{obs}}$ and $h_\mathrm{R_{obs}}$ are
the observed amplitude of left and right polarizations of the GW respectively. Note that GR is recovered if $\kappa$ is 0.
This is a preliminary version of the birefringence model in CS, which assume there is no frequency dependence.
This would create a degeneracy between kappa and iota, as they can affect the amplitude ratio in the same way.

In this paper, we considered the frequency dependence of the birefringence as well.
Thus, the amplitude ratio of the left and right polarisations will be as follow.
\begin{equation}
    \left(\frac{h_\mathrm{L_{obs}}}{h_\mathrm{R_{obs}}}\right)_\mathrm{Biref}^\mathrm{new}=\frac{\exp\left({\kappa\frac{d_C}{1\mathrm{ Gpc}}\frac{f}{100\mathrm{ Hz}}}\right)\left(1-\cos\iota\right)^2}{\exp\left({-\kappa\frac{d_C}{1\mathrm{Gpc}}\frac{f}{100\mathrm{Hz}}}\right)\left(1+\cos\iota\right)^2}\,,
\end{equation}where $f$ is the frequency of the GWs and $\kappa$ is the dimensionless opacity parameter that represent the strength of the birefringence.
This frequency dependence provides stronger enhancement or suppression from the birefringence to high frequency components of the GW, and visa versa.
Also, the frequency dependence can break the degeneracy between kappa and iota,
as this would make the effect of the birefringence affect the amplitude ratio differently compared to iota.

Estimations on $\kappa$ using Bilby with data from the the third LIGO-Virgo catalog, GWTC-3, was performed to constrain the plausible range of $\kappa$.
Bilby is a bayesian toolkit for GW analysis, which could calculate posterior of GW waveform parameters based on signals from
interferometer and priors of the parameters provided. \citep{Ashton_2019}

We assume GWs are produced at the BBH following GR. As the GWs propagate through space,
the effect of the birefringence will be built up as the distance increases.
To implement the amplitude birefringence on Bilby, waveforms are modified according to the following equation:
\begin{equation}
    h_\mathrm{L,R}^{\mathrm{Biref}}=
    h_\mathrm{L,R}^{\mathrm{GR}}\times
    \exp\left(\pm\kappa\frac{d_C}{1\mathrm{ Gpc}}\frac{f}{100\mathrm{ Hz}}\right)\,.
\end{equation}
This modification allows Bilby to perform estimations on $\kappa$, as the waveforms will depends on $\kappa$ as well.

Consider GW150914, the first detected GW by LIGO. In figure \ref{fig:GW150914_corner}, the posteriors of the parameters of the GW150914 is shown.
With the frequency independent birefringence model, the posteriors look different from the posteriors with GR.
This is because there is a degeneracy between $\kappa$ and $\iota$ without the frequency dependence,
so $\iota$ was less constrained compared to the PE with GR.
On the other hand, with the frequency dependent birefringence model, the posteriors look similar to the posteriors with GR.
This is because the degeneracy was broken by the frequency dependence. In this case, the posteriors can recover the PE with GR,
and the most probable value of $\kappa$ is close to 0, which means the birefringence is weak or absent, and GR can be recovered.

\begin{figure*}[h]
    \script{GW150914_corner.py}
    \includegraphics[width=\textwidth]{figures/GW150914_corner.png}
    \caption{The posterior of $\kappa$, luminosity distance $d_L$ and $\cos{\iota}$ for GW150914.
    The three sets of plots are the parameter estimations (PE) done by LIGO with GR \citep{LIGO_2021},
    the PE done by us with the frequency independent birefringence and the frequency dependent birefringence respectively.
    Note that there is no posterior of $\kappa$ for the PE from LIGO, as the LIGO PE is based on GR,
    which does not suggest GWs amplitude birefringence.}
    \label{fig:GW150914_corner}
\end{figure*}

\section{Introduction}



\section{Method}



\section{Results}



\section{Discussion}



\section{Acknowledgements}



\bibliography{bib}

\end{document}