\documentclass[aps,prd,twocolumn,superscriptaddress,preprintnumbers,floatfix,nofootinbib]{revtex4-2}

\usepackage{showyourwork}
\usepackage{amsfonts,amssymb,amsmath}
\usepackage[nolist,nohyperlinks]{acronym}
\usepackage{bookmark}

\newcommand{\infd}{\mathrm{d}}

\newcommand*{\mi}[1]{\textsf{\color{magenta} [\textbf{MAX:} #1]}}
\newcommand*{\wf}[1]{\textsf{\color{cyan} [\textbf{WILL:} #1]}}

\begin{document}

\title{Constraining gravitational wave amplitude birefringence with GWTC-3}

\author{Thomas C. K. Ng}
\email{thomas.ng@link.cuhk.edu.hk}
\affiliation{Department of Physics, The Chinese University of Hong Kong, Shatin, Hong Kong}

\author{Maximiliano Isi}
\email{misi@flatironinstitute.org}
\affiliation{Center for Computational Astrophysics, Flatiron Institute, 162 5th Ave, New York, NY 10010, United States}

\author{Kaze W. K. Wong}
\email{kwong@flatironinstitute.org}
\affiliation{Center for Computational Astrophysics, Flatiron Institute, 162 5th Ave, New York, NY 10010, United States}

\author{Will M. Farr}
\email{wfarr@flatironinstitute.org}
\affiliation{Center for Computational Astrophysics, Flatiron Institute, 162 5th Ave, New York, NY 10010, United States}
\affiliation{Department of Physics and Astronomy, Stony Brook University, Stony Brook NY 11794, United States}

\date{\today}

\begin{abstract}
    The propagation of gravitational waves can reveal fundamental features of the structure of spacetime.
    For instance, differences in the propagation of gravitational-wave polarizations would be a smoking gun for parity violations in the gravitational sector, as expected from birefringent theories like Chern-Simons gravity.
    Here we look for evidence of amplitude birefringence in the latest LIGO-Virgo catalog (GWTC-3) through the use of birefringent templates inspired by dynamical Chern-Simons gravity.
    From 71 binary-black-hole signals, we obtain the most precise constraints on amplitude birefringence yet, an order of magnitude more stringent than previous results.
\end{abstract}

\maketitle

\begin{acronym}
\acro{GW}{gravitational wave}
\acro{GR}{general relativity}
\acro{CBC}{compact-binary coalescence}
\acro{BH}{black hole}
\acro{BBH}{binary black hole}
\acro{LVK}{LIGO-Virgo-KAGRA Collaboration}
\acro{PE}{parameter estimation}
\acro{FAR}{false-alarm rate}
\acro{GWOSC}{the Gravitational Wave Open Science Center}
\acro{SNR}{signal-to-noise ratio}
\end{acronym}

\section{Introduction}
\label{sec:Introduction}
% Motivation
\Ac{GW} detections by the \ac{LVK} \citep{LIGO, Virgo, KAGRA} are now routinely used to test various aspects of Einstein's theory of \ac{GR} \citep{LIGOScientific:2016lio,LIGOScientific:2018dkp,LIGOScientific:2021sio}.
Among those, measurements of the basic properties of \acp{GW}, like their speed and polarization, can directly probe the fundamental symmetries of the underlying theory of gravity \citep{Will:2018bme}.
For instance, unequal propagation of \ac{GW} polarization eigenstates would reveal that spacetime is birefringent, a smoking gun for parity-odd theories like Chern-Simons gravity \citep{Jackiw:2003pm,Alexander:2009tp,Sopuerta:2009iy}.
Here we constrain the magnitude of possible amplitude birefringence using \ac{BBH} signals from the latest \ac{LVK} catalog, GWTC\nobreakdash-3 \citep{GWTC-3}.

% Previous studies
Previous studies have constrained amplitude birefringence by performing different statistical analyses.
\citet{Yamada_2020} and \citet{Wang_2021} both performed \ac{PE} on the events in the first \ac{GW} transient catalog \citep{GWTC-1}, GWTC-1, using birefringent templates.
\citet{Okounkova_2022} considered the distribution of observed inclinations of the \ac{GW} events in the second \ac{GW} transient catalog \citep{GWTC-2}, GWTC-2, to look for signs of birefringence.

% What's new?
In this study, we use a frequency-dependent birefringence model to constrain the strength of \ac{GW} amplitude birefringence by performing \ac{PE} on \ac{LVK} binaries.
This model is a better approximation of the birefringence effect expected from theory than the frequency-independent model used in \citet{Okounkova_2022}.
Compared to other studies, we perform \ac{PE} on more events, including events new to GWTC-3 \citep{GWTC-3}, and use a phenomenology-oriented parametrization.
We consider 71 binary black hole merger events with a \ac{FAR} $\leq1\mathrm{yr^{-1}}$, as listed in Table I of \citet{GWTC-3_population}.
We discuss single-event results in detail, and identify degeneracies between birefringence and spin effects, in addition to the already known correlations with source orientation and distance.
We use the results from individual events to place a collective population constraint on the strength of \ac{GW} amplitude birefringence from GWTC-3.

% Section guide
In Sec.~\ref{sec:Background}, we briefly review the background of \ac{GW} amplitude birefringence.
In Sec.~\ref{sec:Method}, we describe the modification we made to the waveform model, mention the configuration we used in the \ac{PE} and show the method we used to obtain the population constraint on \ac{GW} amplitude birefringence.
In Sec.~\ref{sec:Results}, we present the population constraint on \ac{GW} amplitude birefringence we obtained and show the results of individual \ac{PE}.
In Sec.~\ref{sec:Discussion}, we discuss the limitation of this study and provide suggestions for future studies.

\section{Background}
\label{sec:Background}

\subsection{Birefringence}
\label{sec:waveform}

% GW polarization
In \ac{GR}, \acp{GW} are comprised of two independent polarization modes, usually represented in the linear basis of plus ($+$) and cross ($\times$) states.
In the Fourier domain, these can be combined into left-handed (L) and right-handed (R) circular states (see, e.g., \cite{Isi:2022mbx}),
\begin{equation}
    h_{L/R} = \frac{1}{\sqrt{2}}\left(h_+ \pm i h_\times\right)\,,
\end{equation}
where $h$ is the frequency domain strain, with the plus and minus signs for L and R respectively.
These circular modes represent eigenstates of the helicity operator (helicity $\pm2$) and possess a definite parity.
Einstein's theory, which is parity even, predicts no difference in the dynamics of these two states.

% Waveform modification
Yet, parity odd extensions of \ac{GR} may make distinctions between the two circular polarizations, potentially appearing in both the generation and propagation of \acp{GW}.
The latter can manifest in changes to the relative amplitude and phase of the polarizations that accrue as the wave propagates, giving us hope of detecting initially small effects that compound over long propagation distances.

In particular, \emph{amplitude} birefringence would enhance one polarization mode over the other.
To first order, in theories like Chern-Simons gravity, the Fourier-domain waveform observed a comoving distance $d_C$ away from the source can be written as
\begin{equation}
    h_{L/R}^{\mathrm{br}}(f) =
    h_{L/R}^{\mathrm{GR}}(f) \times
    \exp\left(\pm\kappa\frac{d_C}{1\, \mathrm{Gpc}}\frac{f}{100\,\mathrm{Hz}}\right)\,,
    \label{eq:waveform_modification}
\end{equation}
where the emitted waveform $h_{L/R}^{\mathrm{GR}}$ is modified by an exponential birefringent factor to yield the observed waveform $h_{L/R}^{\mathrm{br}}$.
The overall magnitude of this effect for a given distance and frequency $f$ is set by a dimensionless opacity parameter, $\kappa$, which encodes the intrinsic strength of the birefringence.
The emitted waveform for a given source (i.e., the waveform observed in the near zone, very close to the source) will generally differ from the analogous waveform predicted by \ac{GR} \cite{Alexander:2009tp,Okounkova:2019zjf}; however, since we expect most viable modifications to \ac{GR} to be intrinsically small (e.g., \cite{Okounkova:2022grv}), it is standard to approximate the emitted waveform by the prediction from \ac{GR} (hence the notation ``$h^{\rm GR}$'' above).

Although the intrinsic modification is small, the effect targeted by Eq.~\eqref{eq:waveform_modification} accumulates as the \ac{GW} propagates.
During propagation, the effect of birefringence will be built up with the number of cycles, which is itself a function of the distance traveled and the frequency of the \acp{GW}.
According to Eq.~\eqref{eq:waveform_modification}, a positive $\kappa$ means the left-handed polarization is enhanced over the right-handed polarization, while a negative $\kappa$ means the opposite;
when $\kappa=0$, the observed waveform is the same as \ac{GR} predicts, meaning there is no birefringence.

Equation \eqref{eq:waveform_modification} can be derived as the first order effect in an expansion away from \ac{GR} under multiple frameworks.
In general, $\kappa$ will be a function of the theory parameters and the cosmological history, e.g., the value of the pseudo-scalar field and its derivative in Chern-Simons gravity \cite{Alexander:2009tp}.
Since it originates from a truncated series expansion, Eq.~\eqref{eq:waveform_modification} is a good approximation only for small exponents, 
\begin{equation}
\left|\kappa\right| \left(d_C/1\,\mathrm{Gpc}\right) \left(f/100\, \mathrm{Hz}\right) < 1\, .
\end{equation}
Otherwise, more frequency-dependent terms would enter the exponent of Eq.~\eqref{eq:waveform_modification} in a theory-dependent way.
\mi{specify what the parameter for the expansion is, add citations, and check with a theorist}

\begin{figure}
    \script{birefringence.py}
    \includegraphics[width=\columnwidth]{figures/birefringence.pdf}
    \caption{
        \emph{Illustration of amplitude birefringence.} The GR waveform for the $\ell=|m|=2$ mode of a nonprecessing BBH seen edge-on ($\cos\iota = 0$) is linearly polarized and thus contains equal amounts of left- and right-handed modes for all frequencies (dotted, top).
        However, if spacetime were birefringent following Eq.~\protect\eqref{eq:waveform_modification}, the waveform observed on Earth would contain different fractions of the two circular modes, with higher frequencies affected more strongly (solid, top). 
        In the time domain, this manifests as a time-dependent amplification of the waveform, with a stronger effect at later times when the chirp reaches a higher instantaneous frequency (bottom).
        For this example, the black holes do not spin and have equal masses $m_1 = m_2 = 10\, M_\odot$, and we have chosen a luminosity distance $d_L = 400\, {\rm Mpc}$ and $\kappa = 0.6$.
        }
    \label{fig:birefringence}
\end{figure}

\subsection{Inclination and other degeneracies}
\label{sec:inclination}

% Effect on inclination PE results
Under certain conditions, the effect of birefringence can be degenerate with a change in the orientation of the source with respect to the line of sight \cite{Alexander:2009tp}.
Concretely, for a nonprecessing compact binary inspiral in \ac{GR}, the observed amplitude ratio of the left-handed and right-handed polarizations is only a function of the inclination $\iota$, the angle between the orbital angular momentum of the source and the line of sight \cite{Blanchet:2013haa}.
For  the dominant $\ell = |m| = 2$ angular mode of the radiation, the relation between the amplitude ratio and the inclination is
\begin{equation}
    \frac{h_{L}^\mathrm{GR}}{h^\mathrm{GR}_{R}}=\left(\frac{1-\cos\iota}{1+\cos\iota}\right)^2\,
\end{equation}
for all frequencies (see, e.g., Sec.~IIIC in \cite{Isi:2022mbx}).

Since birefringence impacts the observed amplitude ratio of left- and right-handed modes, it could also affect inferences about the source inclination \cite{Alexander:2009tp}.
However, the two effects are degenerate only if the frequency dependence of Eq.~\eqref{eq:waveform_modification} is neglected.
This is easy to see from Eq.~\eqref{eq:waveform_modification}, since the implied polarization ratio for the $\ell = |m| = 2$ mode of a nonprecessing source is
\begin{equation}
    \frac{h_{L}^\mathrm{br}}{h_{R}^\mathrm{br}}=\left(\frac{1-\cos\iota}{1+\cos\iota}\right)^2
    \exp\left({2\kappa\frac{d_C}{1\, \mathrm{Gpc}}\frac{f}{100\, \mathrm{Hz}}}\right)\, .
    \label{eq:modified_amplitude_ratio}
\end{equation}
For an isolated Fourier mode of definite frequency $f$, the effect of birefringence will be degenerate with a change in inclination; however, if multiple modes come into play, then no change in inclination alone can mask the effect of birefringence, which will affect the time domain waveform nontrivially (Fig.~\ref{fig:birefringence}).

\citet{Okounkova_2022} took the effect of birefringence to be independent of the frequency, which is a zeroth-order approximation of the birefringence model in Chern-Simons gravity.
This assumption results in a full degeneracy between $\kappa$ and $\iota$:
to reconstruct the amplitude ratio from the interferometer data, a value of $\iota$ representing a more face-off inspiral can pair with a positive value of $\kappa$, or a value of $\iota$ representing a more face-on inspiral with a negative value of $\kappa$.
That fact can be used to constrain frequency-independent birefringence by searching for features in the distribution of inferred inclinations \cite{Okounkova_2022}.

By implementing Eq.~\eqref{eq:waveform_modification}, which is a first-order approximation of the birefringence model, we generally break the degeneracy between birefringence and source orientation; this was also the case in the frequency-dependent relations studied in \cite{Yamada_2020,Wang_2021}.
Nevertheless, there exist systems for which the degeneracy cannot be perfectly broken because not enough frequencies are available in the data.
This may be the case for quasimonochromatic sources, like nonaxisymmetric pulsars or very light binaries, which are well approximated by a single Fourier mode.

As we will find in Sec.~\ref{sec:Results}, the effect of birefringence can be (partially) degenerate with other parameters besides source inclination.
Indeed, the frequency-dependent amplification or dampening of polarizations caused by $\kappa$ can sometimes be absorbed by changes in intrinsic parameters, like the spins, with concurrent adjustments to the source inclination and distance.
As a measure of the \ac{BH} spin magnitudes along the orbital angular momentum, we will use the effective spin parameter \cite{Damour:2001tu,Ajith:2009bn,Santamaria:2010yb}
% \begin{equation}
% \chi_{\rm eff} \equiv \frac{1}{M} \left(m_1 \left|\vec{\chi}_{1\parallel}\right| + m_2\left|\vec{\chi}_{2\parallel}\right|\right)
% \end{equation}
\begin{equation}
\chi_{\rm eff} \equiv \frac{1}{1+q} \left(\left|\vec{\chi}_{1\parallel}\right| + q\left|\vec{\chi}_{2\parallel}\right|\right)
\end{equation}
where $|\vec{\chi}_{i\parallel}|$ are the norms of the projections of the dimensionless spin vectors along the orbital angular momentum, and the  mass ratio is $q \equiv m_2/m_1 \leq 1$.
We will study the interplay between birefringence and precession by focusing on the posterior of the precessing spin parameter $\chi_p$, defined as \cite{Schmidt:2014iyl}
\begin{equation}
\chi_p \equiv \max\left\{ |\vec{\chi}_{1\perp}|,\, k |\vec{\chi}_{2\perp}|\right\} ,
\end{equation}
where $|\vec{\chi}_{i\perp}|$ are the norms of the projections of the dimensionless spin vectors onto the orbital plane at a reference time, and $k \equiv q\left(4 q +3\right) / \left(4 + 3q\right)$.
We chart this and other approximate degeneracies as part of the results presented in Sec.~\ref{sec:Results}.

% In particular, we find that, under certain conditions, the frequency-dependent birefringence of Eq.~\eqref{eq:waveform_modification} can mimic the time-dependent amplitude modulation expected from a precessing system.

\section{Method}
\label{sec:Method}

\subsection{Single-event parameter estimation}

To constrain birefringence, we reanalyze events from GWTC-3 \citep{GWTC-2.1, GWTC-3} implementing Eq.~\eqref{eq:waveform_modification} to directly obtain a posterior on $\kappa$ from the strain of each event.
We analyze the 71 \acp{BBH} that were detected with $\mathrm{FAR} < 1/\mathrm{yr}$; to avoid extended computations on longer signals and considering these are generally at closer distances, we do not analyze systems involving neutron stars in this work.
We procure strain data from \ac{GWOSC} \citep{GWOSC}.

We estimate source parameters using a custom version of the \textsc{Bilby} software \citep{Bilby}, modified from the baseline version to apply Eq.~\eqref{eq:waveform_modification} for any \ac{GR} baseline waveform.
We take the \ac{PE} configuration in \citep{GWTC-2.1, GWTC-3, GWTC-2.1_dataset, GWTC-3_dataset} as a starting point, with \textsc{IMRPhenomXPHM} \citep{Pratten:2020ceb} as the reference waveform.
We apply a distance prior corresponding to a uniform distribution over comoving volume and source-frame time, and set the prior on $\kappa$ to be uniform between $-1$ and $1$. 

For GW190521, we increase the maximum distance allowed by the prior to $1.5\times$ the original value in \cite{GWTC-2.1_dataset}, as the birefringence effect results in posterior support at larger distances.
For GW190720, we decrease the analysis segment from 16 s to 8 s, in order to accommodate missing data near the edges of the 16 s segment in Virgo.
Otherwise, the configuration is as in \citep{GWTC-2.1, GWTC-3, GWTC-2.1_dataset, GWTC-3_dataset}.

In order to validate our \ac{PE} implementation, we reproduce the \ac{LVK} \ac{PE} results obtained assuming \ac{GR} by enforcing $\kappa = 0$; this also has the advantage of producing \ac{GR} runs that are directly comparable to our birefringent runs.
All our \ac{PE} results, including the \ac{GR} validation runs, are published in \citet{dataset}.

\subsection{Collective analysis}

\subsubsection{Shared birefringence parameter}

In the most simplified scenarios, birefringence is a property of spacetime that is not intrinsic to any source or region in space. 
Consequently, we should expect $\kappa$ to take the same value for all signals, whether it vanishes or not.
Under this assumption, the posterior on $\kappa$ inferred collectively from all events is simply obtained from the product of the individual likelihoods, $p(d_i \mid\kappa) \propto p(\kappa \mid d_i)/p(\kappa)$, such that
\begin{equation}
    p(\kappa \mid \{d_i\})\propto p(\kappa) \prod_{i}\frac{p(\kappa \mid d_i)}{p(\kappa)}\,,
    \label{eq:restricted_posterior}
\end{equation}
where $d_i$ is the strain data for the $i$\textsuperscript{th} event, and $p(\kappa)$ is the prior on $\kappa$; since the prior is uniform, in our case Eq.~\eqref{eq:restricted_posterior} reduces to the product of the posteriors, namely $p(\kappa \mid \{d_i\}) \propto \prod_{i}p(\kappa \mid d_i)$.
We use Eq.~\eqref{eq:restricted_posterior} to obtain the primary constraint presented in this work.

\subsubsection{Nonshared birefringence parameters}
\label{sec:method:hier}

% Bayesian Hierarchical Modeling
Under many frameworks, birefringence is brokered by an extra parity-odd field that couples to gravity.
In that case, the effective strength of birefringence may vary along the different lines of sight to each event, depending on cosmic history and the local evolution of the field.
The \emph{simplest} assumption is that the field manifests equally for all events, as assumed in the previous subsection, but this is not strictly required.
These considerations motivate a collective analysis that does not assume $\kappa$ is shared across events \cite{Zimmerman:2019wzo,Isi:2022cii}.
Such an analysis has the additional advantage of helping us further characterize our set of measurements, and identify potential outliers.

To do this, we apply hierarchical Bayesian inference \cite{Loredo:2004nn} to model the distribution of $\kappa$'s consistent with the observed data:
we posit that, rather than a unique global value of $\kappa$, there is a specific value of the parameter, $\kappa_i$, associated with each event, and that this is drawn from some unknown distribution of true underlying values; from the imperfect measurements of $\kappa_i$ for each event, we may reconstruct the underlying distribution.
If we are interested in constraining the first two moments of the distribution, it is convenient to parametrize the $\kappa_i$'s as drawn from a Gaussian with unknown mean $\mu$ and variance $\sigma^2$, i.e., $\kappa_i \sim \mathcal{N}(\mu, \sigma^2)$ \cite{Isi:2019asy}, and measure those hyperparameters from the collection of observed data.

If \ac{GR} is correct and there is no birefringence, we should find the observed $\kappa$ distribution to be consistent with a delta function at the origin ($\kappa_i = 0$ for all $i$, or $\mu=\sigma=0$); on the other hand, if spacetime is globally birefringent, we expect to find a delta function at some nonzero value ($\kappa_i = \kappa \neq 0$, or $\mu = \kappa$ and $\sigma=0$).
But this analysis also has the power to reveal unexpected physics or systematics in our measurements: if $\sigma$ is confidently found to be nonzero, this would imply that our set of measurements is statistically unlikely to originate from a unique $\kappa$ value.
This could signal richer physics than is implied by Eq.~\eqref{eq:waveform_modification} or, more prosaically, that there are outliers in our measurements due to mismodeling, e.g., in the waveform approximant or the noise of the detector.

Starting from the posterior on $\kappa$ from each $i$\textsuperscript{th} event, $p(\kappa_i\mid d_i)$, the posterior on the hyperparameters $\mu$ and $\sigma$ can be calculated by
\begin{equation}
    p(\mu,\sigma \mid \{d\})\propto p(\mu,\sigma)\prod_{i}\int\frac{p(\kappa_i\mid d_i)}{p(\kappa_i)}p(\kappa_i\mid\mu,\sigma)\,\infd\kappa_i,
    \label{eq:posterior_of_mu_sigma}
\end{equation}
where $p(\kappa)$ is the prior initially applied to $\kappa$ during \ac{PE}, which in our case is a uniform distribution, $\mathcal{U}[-1,1]$.
Further choosing the hyperpriors on $\mu$ and $\sigma$ to be uniform, Eq.~\eqref{eq:posterior_of_mu_sigma} simplifies to
\begin{equation}
    p(\mu,\sigma\mid\{d\})\propto\prod_{i}\int p(\kappa_i\mid d_i)\, p(\kappa_i\mid\mu,\sigma)\,\infd\kappa_i\,
\end{equation}
where $p(\kappa_i\mid\mu,\sigma) \propto \exp(-|\kappa_i - \mu|^2/2\sigma^2)$ is the usual Gaussian likelihood.

Beyond measuring the population mean and variance, we also calculate the expected population distribution of $\kappa$ marginalized over $\mu$ and $\sigma$.
This is defined by
\begin{equation}
p(\kappa_i \mid \{d\})=\int p(\kappa_i \mid \mu,\sigma)p(\mu,\sigma\mid \{d\})\,\infd\mu\,\infd\sigma\, , 
    \label{eq:generic_posterior}
\end{equation}
and represents our overall expectation for the true values of $\kappa$, given the observed set of individual measurements and, crucially, assuming the underlying distribution is Gaussian.
Although the measurement of the population mean and variance applies irrespective of whether the underlying $\kappa_i$ distribution is truly Gaussian, the specific shape of the population-marginalized distribution of Eq.~\eqref{eq:generic_posterior} is not; therefore, Eq.~\eqref{eq:generic_posterior} should not be interpreted as giving a generic inference on the $\kappa$ distribution---more expressive models than a Gaussian would be better suited to that purpose. 
To sample the posterior distribution of $\mu$ and $\sigma$, we use the sampling package \textsc{flowMC} \citep{flowMC}.

\section{Results}
\label{sec:Results}

In this section, we present the results of our study.
We first show the $\kappa$ measurements from of all events in our set, as well as the resulting global measurement of $\kappa$ that represents our primary constraint on birefringence (Sec.~\ref{sec:results:gwtc}).
We then assess the collection of measurements in more detail through a hierarchical analysis (Sec.~\ref{sec:results:hier}).
Finally, we discuss some special events individually, and outline the degeneracies that arise between birefringence and orbital precession (Sec.~\ref{sec:results:notable}).

\begin{figure}
    \script{violin_kappa.py}
    \includegraphics[width=\columnwidth]{figures/violin_kappa.pdf}
    \caption{
        Individual-event $\kappa$ posteriors (distributions), and joint measurement (blue band, 90\% CI; blue line, median).
    }
    \label{fig:violin_kappa}
\end{figure}

\subsection{GWTC-3 result}
\label{sec:results:gwtc}

% Violin plot
Figure \ref{fig:violin_kappa} shows the main result of our GWTC-3 analysis as represented by the posterior distribution on $\kappa$ (abscissa) obtained individually for each event (ordinate).
Posteriors are colored by the respective posterior mean distance to the origin in units of standard deviation, i.e., $|\mu_i/\sigma_i|$ for each event $i$.\footnote{We use the $i$ notation for individual events here to disambiguate from the population mean $\mu$ and standard deviation $\sigma$ of Eq.~\eqref{eq:posterior_of_mu_sigma}.}
The collective measurement obtained by assuming a shared $\kappa$ across events, Eq.~\eqref{eq:restricted_posterior}, is represented by its 90\%-credible symmetric interval (blue band) around the median (blue line); this joint measurement is fully consistent with $\kappa = 0$ (dashed line) with the credible level of \variable{output/CL_kappa_0.txt}, meaning we find no evidence for birefringence.

Figure \ref{fig:violin_kappa} makes it clear that not all \acp{BBH} in GWTC-3 are equally informative about birefringence.
When considered individually, the events that best constrain $\kappa$ are listed in Table~\ref{tab:best_events_kappa}, in order of increasing standard deviation $\sigma_i$.
That table also shows the credible level (CL) at which the posterior supports $\kappa = 0$, whereby $\mathrm{CL} = 0$ ($\mathrm{CL} = 1$) means the posterior supports that value with high (low) probability.

Judging by $\mu_i/\sigma_i$, the two events that show the largest tension with $\kappa = 0$ are GW170818, for which $\mu_i / \sigma_i =$ \variable{output/GW170818_constraint.txt}, and GW200129\_065458 (henceforth GW200129), for which $\mu_i / \sigma_i =$ \variable{output/GW200129_constraint.txt}.
However, as we discuss in Sec.~\ref{sec:GW200129}, we have reason to think that the preference for $\kappa < 0$ in GW200129 might be driven by noise anomalies in the Virgo detector; with that in mind, in the next section we consider the effect of excluding this event from the joint result (we find its impact to be minimal).

\begin{table}
    \caption{Events that best constrain $\kappa$, sorted by posterior standard deviation $\sigma_i$. CL is the credible level of $\kappa = 0$.}
    \begin{ruledtabular}
        \variable{output/best_events_kappa.txt}
    \end{ruledtabular}
    \label{tab:best_events_kappa}
\end{table}

\begin{table}
    \caption{Events with bimodality in the $\kappa$ posterior, the \ac{GR} measurement of their detector-frame total mass ($M$), their $\chi_p$ measured without (GR) and with (BR) brefringence, and the credible level of $\kappa = 0$ (CL).}
    \begin{ruledtabular}
        \variable{output/bimodal_events_mass.txt}
    \end{ruledtabular}
    \label{tab:bimodal_events_mass}
\end{table}

Finally, a set of events stands out in Fig.~\ref{fig:violin_kappa} due to evident bimodality in the $\kappa$ posterior.
To a varying degree, that is the case for those events listed in Table~\ref{tab:bimodal_events_mass}, which tend to have quite high total masses in the detector frame (Table~\ref{tab:bimodal_events_mass} shows total mass as measured in the standard \ac{GR} analysis).
For these bimodal posteriors, $\mu_i/\sigma_i$ is not a good proxy for agreement with \ac{GR}; instead, we can rely on $\mathrm{CL}(\kappa = 0)$.
By this measure, the bimodal events are some of the least consistent with $\kappa = 0$, GW190521 in particular.
% (Nevertheless, configurations with $\kappa = 0$ for this event are still within the higher dimensional 90\%-credible region, when other parameters are considered and not just the one-dimensional marginal; we discuss this in Sec.~\ref{sec:GW190521}.)

As we anticipated in Sec.~\ref{sec:inclination}, we understand the bimodality in $\kappa$ to be linked to spin effects, and often to precessing morphologies in particular.
Other parameter degeneracies also come into play, especially for the lighter events GW191105\_143521 (henceforth GW191105) and GW170104.
We discuss this further in a dedicated section below (Sec.~\ref{sec:results:notable}).

\begin{figure}
    \script{corner_Gaussian.py}
    \includegraphics[width=\columnwidth]{figures/corner_Gaussian.pdf}
    \caption{
        The posterior of the $\kappa$ population hyperparameters $\mu$ and $\sigma$, including (blue) and excluding (orange) GW200129 from the collection of events.
        The 2D contours correspond to the $39.35\%$ and $90\%$ credible levels.
        The plot shows that the population constraint on $\kappa$ is consistent with no birefringence ($\mu=\sigma=0$) at the 90\% credible level.
    }
    \label{fig:corner_Gaussian}
\end{figure}

\subsection{Hierarchical modeling}
\label{sec:results:hier}

Figure \ref{fig:violin_kappa} shows a certain degree of variance in the distribution of $\kappa$ posteriors for different events, including some apparent outliers like GW170818 or GW190521.
This is not unexpected: assuming independent Gaussian noise instantiations for each event, we might expect up to ${\sim}3$ out of the 71 posteriors (i.e., ${\sim}5\%$) to deviate away from the true $\kappa$ value by over ${\sim}2\sigma$ due to noise alone.

To further assess the statistical properties of the set of posteriors in Fig.~\ref{fig:violin_kappa}, we apply the hierarchical analysis described in Sec.~\ref{sec:method:hier}.
By characterizing the population mean and standard deviation over events, this also allows us to obtain a collective measurement that does not assume all events share the same value of $\kappa$.

We summarize the result of this exercise in Fig.~\ref{fig:corner_Gaussian}, which shows the posterior on the population mean $\mu$ and standard deviation $\sigma$ inferred from the collection of observations in Fig.~\ref{fig:violin_kappa}.
The figure shows two distributions, which result from analyses with (blue) and without (orange) the potentially-contaminated event GW200129; the difference between the two is mainly restricted to a slight shift in $\mu$, indicating that GW200129 has a small effect on our overall population inference.

Both versions of the posterior support the lack of birefringence ($\mu = \sigma = 0$) within 90\% credibility; from the marginals of the result including all events, we constrain $\mu =$ \variable{output/mu_median.txt} for 90\%-credible symmetric intervals around the median, and $\sigma <$ \variable{output/sigma_median.txt} for the 90\%-credible one-sided upper limit.
However, even though $\sigma = 0$ is well supported, the $\sigma$ posterior peaks visibly away from the origin, indicating some preference for a nonzero variance.
This could be a sign of the presence of outliers in our sample.

As a visual check for outliers, we reconsider the set of measurements in Fig.~\ref{fig:violin_kappa} in light of the hierarchical result for $\mu$ and $\sigma$ in Fig.~\ref{fig:corner_Gaussian}, including GW200129 (blue curve).
This amounts to reweighting the $\kappa$ posterior for each event under a population prior marginalized over $\mu$ and $\sigma$, conditional on the measurements from all other events \cite{Callister:T2100301}.
Figure \ref{fig:corner_Gaussian} does not show evidence for any of the events being in obvious tension with the population, even though the GW170818 curve stands out from the rest due to its higher support for $\kappa > 0$.
This feature appears to offset a few other events which tend to favor $\kappa < 0$.
The interaction between these distributions leads to a hyperposterior that is fully consistent with $\mu = 0$ while offering some support for $\sigma > 0$ (Fig.~\ref{fig:corner_Gaussian}).
Future observations will determine whether there is truly evidence for a nonvanishing variance in this population.

\begin{figure}
    \script{reweighed_kappa.py}
    \includegraphics[width=\columnwidth]{figures/reweighed_kappa.pdf}
    \caption{
        Individual-event $\kappa_i$ distributions of Fig.~\ref{fig:violin_kappa} reweighted in light of the population-level inference of Fig.~\ref{fig:corner_Gaussian}, marginalizing over $\mu$ and $\sigma$.
        Curves are colored by the magnitude of the supported deviation, as represented by the respective posterior $|\mu_i / \sigma_i|$; we label the events most in tension with $\kappa_i = 0$ by that same measure.
    }
    \label{fig:reweighted_kappa}
\end{figure}

% Constraint on $\kappa$
Finally, the hierarchical result in Fig.~\ref{fig:corner_Gaussian} can be translated into an overall expectation for $\kappa_i$ under the assumption of a Gaussian distribution via Eq.~\eqref{eq:generic_posterior}.
We show the result of doing this in Fig.~\ref{fig:posterior_kappa}, where we also compare to the posterior on $\kappa$ obtained by assuming a shared value across events (same result shown as a blue band in Fig.~\ref{fig:violin_kappa}).
The hierarchical measurement leads to an expectation that $\kappa_i =$ \variable{output/generic_kappa_median.txt}, whereas the shared-$\kappa$ measurement implies $\kappa =$ \variable{output/restricted_kappa_median.txt}, both symmetric 90\%-credible intervals around the median.  

\begin{figure}
    \script{posterior_kappa.py}
    \includegraphics[width=\columnwidth]{figures/posterior_kappa.pdf}
    \caption{
        The generic and restricted posterior of $\kappa$.
        The blue solid and dashed lines show the restricted posterior of $\kappa$ with and without GW200129, respectively.
        The orange solid and dashed lines show the generic posterior of $\kappa$ with and without GW200129, respectively.
        The black dashed line marks the absence of birefringence ($\kappa=0$).
    }
    \label{fig:posterior_kappa}
\end{figure}

% Case studies
\subsection{Notable events}
\label{sec:results:notable}

Having established that the collection of detections is globally consistent with $\kappa=0$, here we focus on four events whose $\kappa$ posteriors stand out in Fig.~\ref{fig:posterior_kappa}: GW170818, GW200129, GW190521 and GW191105.
When considered in isolation, the first of these is the unimodal event with the most significant support for nonzero $\kappa$; the second shows signs of potential noise anomalies; the third is representative of a class of heavy events with strongly bimodal $\kappa$ posteriors; and the fourth of a smaller class of lighter events with slightly bimodal posteriors.
Through these examples, we elucidate the interaction between $\kappa$ and the source luminosity distance, inclination and spin precession.

% Case: GW170818
\subsubsection{GW170818}
\label{sec:GW170818}

\begin{figure}
    \script{corner_GW170818.py}
    \includegraphics[width=\columnwidth]{figures/corner_GW170818.pdf}
    \caption{
        GW170818 posterior on $\kappa$, luminosity distance $d_L$ and inclination $\cos\iota$ from our birefringence analysis (blue), compared to the GR result (orange).
        The top right panel shows the marginalized posterior on $\chi_p$: allowing for birefringence reduces the preference for precession.
        (See Fig.~\ref{fig:corner_GW170818_appendix} for a full corner plot.)
    }
    \label{fig:corner_GW170818}
\end{figure}

GW170818 produced the posterior most displaced from $\kappa=0$, when judged by $\mu_i/\sigma_i$ in Fig.~\ref{fig:violin_kappa}.
Figure \ref{fig:corner_GW170818} shows that this happens because birefringence opens up a region of parameter space with $\kappa>0$ for larger distances and smaller inclination angles than would be allowed in the GR case.
We can make sense of this by noting that an edge-on, nonprecessing source produces linearly-polarized waves, meaning that a smaller inclination leads the two circular polarizations to have similar amplitudes.
On the other hand, having $\kappa > 0$ enhances the left-handed modes during propagation, per Eq.~\eqref{eq:waveform_modification}.
The two effects can be balanced to yield the correct polarization ratio observed at the detector (predominantly left-handed, per the preference for $\cos\iota \approx -1$ in the \ac{GR} analysis), as long as the distance is also enhanced to yield the right amount of birefringence and overall signal power.
This is similar to the degeneracy mentioned in Sec.~\ref{sec:inclination}.
% Even though the degeneracy is broken by the frequency dependence to a large extent, it is still possible for getting an extra region with a nonzero $\kappa$.
% This is because there is still a significant likelihood in the region of $\kappa$ close to zero.
% We can see this effect in other events as well, but not as strongly as in GW170818.

It is difficult to unequivocally identify a specific feature of the GW170818 data that leads to this posterior structure.
However, it appears to be related to this event's support for precession, in conjunction with its uncommonly definite measurement of the polarization and spin angles \cite{Varma:2021csh}.
The relevance of precession is evident from the posterior on $\chi_p$ (Fig.~\ref{fig:corner_GW170818}, top right): allowing for $\kappa \neq 0$ leads to reduced support for precession.
We can understand this in reference to Fig.~\ref{fig:birefringence}: if only a short portion of the signal is observed, then the frequency-dependent signal enhancement or dampening due to birefringence can mimic the time-dependent amplitude modulation produced by a precession cycle.
Therefore, under these circumstances, similar morphologies can be obtained by setting $\chi_p > 0$ or $\kappa > 0$, as long as the distance and inclination can also be adjusted accordingly.

The fact that the birefringent analysis favors a high $\kappa$ rather than a high $\chi_p$ is likely a consequence of prior volume: many more configurations are available with long distances and high $\kappa$ than with short distances and small $|\kappa|$.
The preference for $\kappa > 0$ over $\kappa < 0$ (and, therefore, the lack of bimodality in the $\cos\iota$ and $\kappa$ posteriors), is likely related to both the definite measurement of left-handed polarizations and the specific phasing of the precession cycle.
The latter manifests as a precise constraint on the spin orientation and phase angles in the \ac{GR} analysis \cite{Varma:2021csh} (see also Appendix \ref{sec:corner_GW170818_appendix}).
The observed amplitude modulation (say, increasing vs decreasing towards the merger) likely determines the allowed sign of $\kappa$ for this event.
Indeed, the inferred polarization angle in the birefringent case is notably different from the \ac{GR} one.
We provide the full corner plot for all relevant parameters in Fig.~\ref{fig:corner_GW170818_appendix} in Appendix~\ref{sec:corner_GW170818_appendix}.

% Case: GW190521 (massive BBH)
\subsubsection{GW190521}
\label{sec:GW190521}

GW190521 is the most extreme representative of a class of events with bimodal $\kappa$ posteriors (Table \ref{tab:bimodal_events_mass}).
This appears to arise out of a confluence of factors related to this event's high mass and support for precession: GW190521 being the most massive event in our set, its signal is only in-band for a very short time before merger, meaning that frequency-dependent birefringence can more easily resemble spin effects.
In particular, a precession cycle can be approximated by a monotonic amplitude modulation for a short time.
This effect gives rise to a similar degeneracy between $\kappa$ and $\chi_p$ as outlined for GW170818, wherein the prior volume favored configurations with $|\kappa| >0$.
However, in the case of GW190521, the polarization measurement at the detectors is not accurate enough to fully disambiguate between right- and left-handed states (as implied by the bimodal $\cos\iota$ posterior obtained assuming \ac{GR} in Fig.~\ref{fig:corner_GW190521}).
As a result, the GW190521 posterior does not favor $\kappa > 0$ exclusively, but supports both $\kappa > 0$ and $\kappa < 0$, leading to a bimodal posterior.

\begin{figure}
    \script{corner_GW190521.py}
    \includegraphics[width=\columnwidth]{figures/corner_GW190521.pdf}
    \caption{
        GW190521 posterior on $\kappa$, luminosity distance $d_L$ and inclination $\cos\iota$ from our birefringence analysis (blue), compared to the GR result (orange).
        The marginalized posterior on $\chi_p$ is shown in the top right panel.
    }
    \label{fig:corner_GW190521}
\end{figure}

% % Case: GW150914 (example of broken degeneracy)
% \subsubsection{GW150914}
% In this study, we included frequency dependence of birefringence, which would affect the posterior of $\kappa$ obtained from \ac{PE}.
% Consider GW150914, the first GW detected by LIGO, as an example.
% In Fig.~\ref{fig:corner_GW150914}, we show the posteriors of the parameters of GW150914.
% With the frequency-independent birefringence model, the posteriors for $\cos\iota$ look different from the posteriors assuming GR.
% This is because, for a nonprecessing system, there is a degeneracy between $\kappa$ and $\iota$ if the frequency dependence is not included.
% % explain the kappa gap
% 
% On the other hand, with the frequency-dependent birefringence model, the posterior looks similar to the GR posterior.
% This is because the frequency dependence broke the degeneracy, as the effect of birefringence will differ from the effect of changing $\iota$.
% In this case, the most probable value of $\kappa$ is close to $0$, which means the birefringence is weak or absent.
% Therefore, the \ac{PE} result with frequency dependence is consistent with GR.
% 
% \begin{figure}
%     \script{corner_GW150914.py}
%     \includegraphics[width=\columnwidth]{figures/corner_GW150914.pdf}
%     \caption{
%         GW150914 posterior on $\kappa$, luminosity distance $d_L$ and inclination $\cos\iota$ from our frequency-dependent (blue), compared to the GR result (green).
%         The marginalized posterior on $\chi_p$ is shown in the top right panel.
%     }
%     \label{fig:corner_GW150914}
% \end{figure}

% Case: GW191105
\subsubsection{GW191105}
\label{sec:GW191105}

The situation is similar for GW191105 (Fig.~\ref{fig:corner_GW191105}).

\begin{figure}
    \script{corner_GW191105.py}
    \includegraphics[width=\columnwidth]{figures/corner_GW191105.pdf}
    \caption{
        GW191105 posterior on $\kappa$, luminosity distance $d_L$ and inclination $\cos\iota$ from our birefringence analysis (blue), compared to the GR result (orange).
        The posterior on $\chi_{eff}$ and $\kappa$ from our birefringence analysis is shown in the top right panel.
    }
    \label{fig:corner_GW191105}
\end{figure}


% Case: GW200129 (glitch in Virgo data)
\subsubsection{GW200129}
\label{sec:GW200129}

GW200129 is the event with the second largest $|\mu_i/\sigma_i|$ in Fig.~\ref{fig:violin_kappa}.
However, data for this event were affected by a non-Gaussian noise disturbance (glitch) in the Virgo instrument, which was subtracted from the publicly-available data used for parameter estimation \cite{Davis:2022ird}.
Since previous work suggests the degree of glitch subtraction affects the inference for this event \citep{GW200129_glitch}, we consider whether the apparent preference for $\kappa < 0$ could also be tied to the instrumental artifact.

To this end, we perform three additional \ac{PE} runs for GW200129, considering only two detectors at a time: LIGO Hanford and Virgo (HL), LIGO Livingston and Virgo (LV), and LIGO Hanford and LIGO Livingston (HL).
If the preference for $\kappa < 0$ is tied to the glitch in Virgo, we expect it to disappear in the HL run, which excludes Virgo data.

This is indeed the case, as we show in Fig.~\ref{fig:corner_GW200129}: all runs including Virgo lean towards $\kappa < 0$ (solid curves in color), whereas the LIGO-only run is fully consistent with $\kappa = 0$ (dashed black).
While this is not conclusive proof that the glitch itself is driving the result, it does indicate that the Virgo data play a key role in the inference of $\kappa$.
Since more work would be needed to understand the effect of the glitch, this motivates us to consider the effect of excluding this event from the collective analyses above.

\begin{figure}
    \script{corner_GW200129.py}
    \includegraphics[width=\columnwidth]{figures/corner_GW200129.pdf}
    \caption{
        GW200129 posterior on $\kappa$, luminosity distance $d_L$ and inclination $\cos{\iota}$, including different sets of detectors in the analysis per the legend.
        The marginalized posterior on $\chi_p$ is shown in the top right panel.
        The main run with all three detectors (HLV, filled blue), shows a preference for $\kappa < 0$, as in Fig.~\ref{fig:violin_kappa}; this preference is more pronounced for two-detector runs that include Virgo (HV and LV, orange and green); however, it disappears if we remove Virgo (HL, dashed black).
        The 2D contours correspond to the $90\%$ credible level.
    }
    \label{fig:corner_GW200129}
\end{figure}

\section{Discussion}
\label{sec:Discussion}

% Comparison with previous studies
\subsection{Comparison with previous studies}
\citet{Okounkova_2022} gave a constraint on GW amplitude birefringence by performing statistical analysis on GWTC-2.
We convert the constraint they gave to the same units as ours to make a comparison.
They were able to constrain $\kappa$ to be $|\kappa| \lesssim 0.74$ at $1 \sigma$.
We obtained a tighter constraint on $\kappa$ with $|\kappa| \lesssim 0.04$ at $1 \sigma$.
Our result is an order of magnitude improvement in constraining $\kappa$.
The main reason is that we have more events from GWTC-3 than GWTC-2.

\citet{Wang_2021} performed \ac{PE} on GWTC-1 events with a frequency-dependent birefringence model.
They formulated the birefringence effect as corrections to the GW waveform on a linear basis.
We convert their constraint to the same units as ours with the derivation in Appendix~\ref{sec:M_PV_derivation}.  \wf{Can you give the answer here (as well as in Appendix B)?}
% Discussion on the difference between our work and theirs

% Future work
\subsection{Future work}

\mi{WIP}
If our inference on $\kappa$ is tied to precession and the spin orientations, we might be susceptible to systematics in the modeling of spin angles in \textsc{IMRPhenomXPHM} and might thus benefit from further analysis with other waveforms like \textsc{NRSur7dq4} \cite{Varma:2018mmi}.
In any case, as we argued in Sec.~\ref{sec:results:hier}, there is no evidence at the moment that this event is inconsistent with the either collection of measurements or with $\kappa = 0$ (Fig.~\ref{fig:reweighted_kappa}).

% BNS
Future work is to perform \ac{PE} on binary neutron star mergers, such as GW170817.
The frequency range of the signal is much wider compared to the binary black hole mergers.
Thus, the difference in the effect of birefringence at different frequencies within the range can be more significant.
The \ac{PE} on binary neutron star mergers can allow us to further constrain the birefringence effect and the beyond-GR theories that predict it.
However, performing \ac{PE} on binary neutron star mergers requires much more computational resources.
Therefore, we may need to wait for future \ac{PE} methods and tools to further our work.

% More observations with higher SNR
Another future work is to apply the same method to more \ac{GW} events and events with a higher \ac{SNR}.
\ac{LVK} will release more events with higher \ac{SNR} in the future.
Using data with higher \ac{SNR} allows us to obtain more precise \ac{PE} results and constrain the birefringence effect more precisely.
And using data from more events will allow us to calculate a more constrained population posterior of $\kappa$.

\begin{acknowledgments}
M.~I., K.~W.~K.~W.~, and W.~M.~F.~ are funded by the Center for Computational Astrophysics at the Flatiron Institute.
The Flatiron Institute provided the computational resources used in this work.

This research has made use of data or software obtained from the Gravitational Wave Open Science Center (gwosc.org), a service of LIGO Laboratory, the LIGO Scientific Collaboration, the Virgo Collaboration, and KAGRA.
LIGO Laboratory and Advanced LIGO are funded by the United States National Science Foundation (NSF) as well as the Science and Technology Facilities Council (STFC) of the United Kingdom, the Max-Planck-Society (MPS), and the State of Niedersachsen/Germany for support of the construction of Advanced LIGO and construction and operation of the GEO600 detector.
Additional support for Advanced LIGO was provided by the Australian Research Council.
Virgo is funded, through the European Gravitational Observatory (EGO), by the French Centre National de Recherche Scientifique (CNRS), the Italian Istituto Nazionale di Fisica Nucleare (INFN) and the Dutch Nikhef, with contributions by institutions from Belgium, Germany, Greece, Hungary, Ireland, Japan, Monaco, Poland, Portugal, Spain.
KAGRA is supported by Ministry of Education, Culture, Sports, Science and Technology (MEXT), Japan Society for the Promotion of Science (JSPS) in Japan; National Research Foundation (NRF) and Ministry of Science and ICT (MSIT) in Korea; Academia Sinica (AS) and National Science and Technology Council (NSTC) in Taiwan.
\end{acknowledgments}

\appendix

\section{Extended corner plot for GW170818}
\label{sec:corner_GW170818_appendix}

Here we present the measurement on all parameters we consider relevant for GW170818, of which Fig.~\ref{fig:corner_GW170818} in the main text represents a subset.
The regular \ac{GR} analysis (GR; orange) and the birefringent analysis (BR; blue) both show notable features.

The GR analysis stands out for its relatively confident identification of the spin angles, $\theta_{1/2}$ and $\phi_{JL}$, as well as the phase and polarization angles, $\phi_{\rm ref}$ and $\psi$.
Of the former, the two $\theta_{1/2}$ parameters encode the tilts of the component \acp{BH} with respect to the orbital angular momentum, $\vec{L}$, while $\phi_{JL}$ is the angle between the projections of $\vec{L}$ and the total angular momentum, $\vec{J}$, along the orbital plane;
of the latter, $\phi_{\rm ref}$ is an overall reference phase, and $\psi$ encodes the orientation of the binary within the plane of the sky \cite{Isi:2022mbx}.
All these parameters are anchored to a reference point in the evolution of the inspiral, which in this case corresponds to the time at which the dominant mode of the observed \ac{GW} signal reaches 20 Hz at the detector (spin angles may be better identified by using a more physical reference point \cite{Varma:2021csh}).
It is unusual for these angles to be well constrained, which suggests that this event displays a particular phase and polarization signature.

The joint posterior on the spin magnitudes, $\chi_1$ and $\chi_2$, indicates that at least one of the component \acp{BH} must have been highly spinning according to this \ac{GR} waveform, as evidenced by the lack of support for $\chi_1=\chi_2 = 0$ (this is not apparent from the marginals because the \acp{BH} were equal in mass \cite{Biscoveanu:2020are}).
Furthermore, the fact that the tilts are favored to be close to $\theta_{1/2} \approx \pi/2$ implies that the spins must lie along the orbital plane.
This is a restatement of the support for high $\chi_p$ in Fig.~\ref{fig:corner_GW170818}.

There are several differences between the birefringent analysis (BR; blue) and the baseline \ac{GR} result (orange).
The fact that the former has a reduced support for precession relative to the latter, as shown in Fig.~\ref{fig:corner_GW170818}, can here be seen in the fact that the BR analysis favors weaker spin magnitudes, $\chi_{1/2}$.

\begin{figure*}[h]
    \script{corner_GW170818_appendix.py}
    \includegraphics[width=\textwidth]{figures/corner_GW170818_appendix.pdf}
    \caption{
        Extended corner plot for GW170818: a supplement to Fig.~\ref{fig:corner_GW170818} discussed in Appendix~\ref{sec:corner_GW170818_appendix}.
    }
    \label{fig:corner_GW170818_appendix}
\end{figure*}

\section{Relation between $\kappa$ and $M_{PV}$}
\label{sec:M_PV_derivation}

\citet{Wang_2021} formulated the birefringence effect as
\begin{equation}
    h_{+/\times}^{PV}(f) = h_{+/\times}^{GR}(f)\mp h_{\times/+}^{GR}(f)(i\delta h-\delta\Psi)\,,
\end{equation}
where $h_{+/\times}^{PV}(f)$ is the modified \ac{GW} waveform in the linear basis, $h_{+/\times}^{GR}(f)$ is the \ac{GW} waveform in the linear basis as \ac{GR} predicts, $\delta h$ is the amplitude modification, and $\delta\Psi$ is the phase modification.
We can rewrite the equation as
\begin{equation}
\begin{split}
    h_{L/R}^{PV}(f)&=h_{L/R}^{GR}(f)(1\mp\delta h\mp i\delta\Psi)\\
    &\approx h_{L/R}^{GR}(f)\exp\left(\mp\delta h\mp i\delta\Psi\right)
\end{split}
\,,
\end{equation}
where $h_{L/R}^{PV}(f)$ is the modified \ac{GW} waveform in the circular basis, $h_{L/R}^{GR}(f)$ is the \ac{GW} waveform in the circular basis as \ac{GR} predicts.
This equation applies to any birefringence model that allows the assumption that the modification to \ac{GR} is small when observed very near the source.
To make a comparison with our work, we can choose $\delta\Psi=0$, which is a case that \citet{Wang_2021} considered.
In their work, they parametrized $\delta h=-A_\nu\pi f$, while
\begin{equation}
    A_\nu=M_{PV}^{-1}(\alpha_\nu(z=0)-\alpha_\nu(z)(1+z))\,,
\end{equation}
where $A_\nu$ is the parity-violating parameter for amplitude birefringence, $M_{PV}$ is the characteristic energy scale, and $\alpha_\nu$ is a arbitrary functions determined by the birefringence model.
They then chose $\alpha_\nu=1$.
Therefore, we can rewrite the equation as
\begin{equation}
    h_{L/R}^{PV}(f)= h_{L/R}^{GR}(f)\exp\left(\mp\frac{\pi zf}{M_{PV}}\right)\,.
\end{equation}
This equation takes the same form as Eq.~\eqref{eq:waveform_modification}.
We can then find the relation between $M_{PV}$ and $\kappa$ by
\begin{equation}
    \left|\mp\frac{\pi zf}{M_{PV}}\right|=\left|\pm\kappa\frac{d_C}{1\,\mathrm{Gpc}}\frac{f}{100\,\mathrm{Hz}}\right|\,.
\end{equation}
Using $H_0d_C=cz$, where $H_0$ is the Hubble constant and $c$ is the speed of light, we can rewrite the equation as
\begin{equation}
    \kappa=\frac{\pi H_0}{c}\left(1\,\mathrm{Gpc}\right)\left(100\,\mathrm{Hz}\right)M_{PV}^{-1}\,.
\end{equation}

\bibliography{bib}

\end{document}
