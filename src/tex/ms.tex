\documentclass[twocolumn]{aastex631}
\usepackage{showyourwork}
\usepackage{amsfonts,amssymb,amsmath}

\begin{document}

\title{Constraining gravitational wave amplitude birefringence with GWTC-3}

\author{Thomas C.K. Ng}

\begin{abstract}
    
\end{abstract}

\section{Everything}
Since Einstein proposed his theory of general relativity (GR), it was tested in a wide range of length scales.
After a century, we know that GR do not agree with quantum theories at some length scale.
To unify both theories, we need to study the possibility of different beyond-GR theories.
Some beyond-GR theories suggested that gravitational waves (GW) amplitude birefringence is a property of space, which GR does not included.

GW consist of two polarizations (i.e. $+$ and $\times$), which could be transformed into two circular polarizations (i.e. left and right) by $h_{\mathrm{R}, \mathrm{L}} = h_+ \pm i h_\times$.
In GR, the amplitude ratio of the left and right polarisations only depends on $\iota$, the angle between our line of sight and the orbital angular momentum of the binary black holes (BH).
\begin{equation}
    \left(\frac{h_\mathrm{L}}{h_\mathrm{R}}\right)_\mathrm{GR}=\left(\frac{1-\cos\iota}{1+\cos\iota}\right)^2\,,
\end{equation}where $h_L$ and $h_R$ are the amplitude of left and right polarizations of the GW respectively.
For the birefringence property mentioned in \citet{Maria_2021}, the amplitude ratio depends on not only $\iota$, but also the distance to the merger and the strength of the birefringence.
\begin{equation}
    \left(\frac{h_\mathrm{L_{obs}}}{h_\mathrm{R_{obs}}}\right)_\mathrm{Biref}^\mathrm{old}=\frac{e^{d_C\widetilde{\kappa}}\left(1-\cos\iota\right)^2}{e^{-d_C\widetilde{\kappa}}\left(1+\cos\iota\right)^2}\,,
\end{equation}where $d_C$ is the comoving distance to the merger, $\widetilde{\kappa}$ is the opacity parameter that represent the strength of the birefringence with units of $L^{-1}$, and $h_\mathrm{L_{obs}}$ and $h_\mathrm{R_{obs}}$ are the observed amplitude of left and right polarizations of the GW respectively.
In this paper, we considered the frequency dependence of the birefringence as well. Thus, the amplitude ratio of the left and right polarisations will be as follow.
\begin{equation}
    \left(\frac{h_\mathrm{L_{obs}}}{h_\mathrm{R_{obs}}}\right)_\mathrm{Biref}^\mathrm{new}=\frac{\exp\left({\kappa\frac{d_C}{1\mathrm{ Gpc}}\frac{f}{100\mathrm{ Hz}}}\right)\left(1-\cos\iota\right)^2}{\exp\left({-\kappa\frac{d_C}{1\mathrm{Gpc}}\frac{f}{100\mathrm{Hz}}}\right)\left(1+\cos\iota\right)^2}\,,
\end{equation}where $f$ is the frequency of the GW and $\kappa$ is the dimensionless opacity parameter that represent the strength of the birefringence.
This frequency dependence provides stronger enhancement or suppression from the birefrigence to high frequency components of the GW, and visa versa.

We performed estimations on $\kappa$ using Bilby with data from the GWTC catalog to constrain the plausible range of $\kappa$.
Bilby is a bayesian toolkit for GW analysis, which could calculate posterior of GW waveform parameters based on signals from interferometer and priors of the parameters provided. \citep{Ashton_2019}
It starts by generating a waveform using the chosen approximant such as IMRPhenomPv2 with a set of parameters.
Bilby will then compute the likelihood of getting the provided data from the interferometer if the waveform just generated is the true waveform.
Using the Bayes' theorem, the posterior probability of the parameters is calculated.
To obtain the whole posterior probability, a chosen sampler is used to run over the parameter space instead of calculating the posterior for the whole parameter space.

To implement the amplitude birefringence on Bilby, waveforms generated by different approximants are modified before calculating the likelihood according to the following equation:
\begin{equation}
    h_\mathrm{L,R}^{\mathrm{new}}=
    h_\mathrm{L,R}^{\mathrm{old}}\times
    \exp\left(\pm\kappa\frac{d_C}{1\mathrm{ Gpc}}\frac{f}{100\mathrm{ Hz}}\right)\,.
\end{equation}
This modification allows Bilby to perform estimations on $\kappa$, as the waveforms generated will depends on $\kappa$ as well.

GWTC-3 Results

Limitation\\
orientation
sensitivity

Future Study\\
Data from more detectors

\section{Introduction}



\section{Method}



\section{Results}



\section{Discussion}



\section{Acknowledgements}



\bibliography{bib}

\end{document}