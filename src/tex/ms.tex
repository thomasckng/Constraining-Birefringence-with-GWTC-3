\documentclass[twocolumn]{aastex631}
\usepackage{showyourwork}
\usepackage{amsfonts,amssymb,amsmath}

\begin{document}

\title{Constraining gravitational wave amplitude birefringence and Chern-Simons gravity with GWTC-3}

\author{Thomas C.K. Ng}

\begin{abstract}
    
\end{abstract}

\section{Introduction}

Beyond-GR theory\\
Birefringence\\
\citep{Maria_2021}

Bilby\\
Bayesian

\section{Method}

Mentioned above, the amount of amplitude birefringence is represented by the opacity parameter $\kappa$.
We performed estimation on $\kappa$ by Bilby with data from the GWTC catalog to constrain the plausible range of $\kappa$.

To implement the amplitude birefringence on Bilby, waveforms generated by different approximants are modified according to the following equation:
\begin{equation}
    h_{L,R}^{\textrm{new}}\left(f\right)=
    h_{L,R}^{\textrm{old}}\left(f\right)\times
    \exp\left(\pm\kappa\frac{d_C}{1\textrm{ Gpc}}\frac{f}{0.1\textrm{ kHz}}\right)
\end{equation}, where $h_{L,R}$ is the amplitude of the left or right polarization of the generated waveform
, $\kappa$ is the dimensionless opacity parameter, $d_C$ is the comoving distance, and $f$ is the frequency of the generated waveform.

Bilby will then perform estimation on the original gravitational wave parameters as well as $\kappa$.

\section{Results}

GWTC-3

\section{Discussion}

Limitation\\
orientation
sensitivity

Future Study\\
Data from more detectors

\section{Acknowledgements}



\bibliography{bib}

\end{document}